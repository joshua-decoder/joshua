\chapter{Grammar Extraction}
\label{ch:grammar-extraction}

Start by compiling an aligned parallel corpus into a directory containing binary data files used by Joshua to represent memory-mapped data structures:

\begin{verbatim}
java -Dfile.encoding=UTF8 -Xms1024m -Xmx1024m \
    -cp bin joshua.corpus.suffix_array.Compile \
    corpus.src corpus.tgt alignments.src-tgt compiled-corpus.josh
\end{verbatim}

Next, select a test set for which rules will be extracted, and run grammar extraction for that test set:

\begin{verbatim}
java -Dfile.encoding=UTF8 -Xms4096m -Xmx4096m  \
    -cp bin joshua.prefix_tree.ExtractRules \
    compiled-corpus.josh test.grammar test.src
\end{verbatim}


\section{Grammar Format}

The grammar extracted by Joshua is a synchronous context free grammar. Each rule consists of a left-hand side nonterminal symbol, a source language right-hand side, a target language right-hand side, and feature values. The source language right-hand side consists of source language terminals, nonterminals, or a mix of terminals and nonterminals. The target language right-hand side likewise consists of target language terminals, nonterminals, or a mix of terminals and nonterminals. The major components of the rule (left-hand side, source right-hand side, target right-hand side, feature value collection) are delineated by three vertical bars $|||$. Any nonterminals that appear on the right-hand sides of the rules are co-indexed.

The feature values currently included in the Joshua grammar rules are, in order:

  \begin{itemize}
  \item Translation model probability $p(e|f)$
  \item Reverse lexical translation model probability $lex p(f|e)$
  \item Lexical translation model probability $lex p(e|f)$
  \end{itemize}

An example grammar rule is shown below\footnote{In the grammar file, the above example rule would appear on only one line, not two. The characters $\backslash\backslash$ are used simply to indicate where the contents of the line continue onto the next line (because this page is not wide enough to display the text on one line).}:

\begin{verbatim}
[X] ||| am meisten [X,1] bei [X,2] ||| most [X,1] for [X,2] ||| \\
          0.47712126 3.9386206 1.8700459
\end{verbatim}






\section{Compare Grammars}

Once a grammar has been extracted, it may be useful to compare it to another grammar. A typical use case may be comparing two grammars that were extracted from the same corpus by different tools.

\begin{verbatim}
java -Dfile.encoding=UTF8 -Xms4096m -Xmx4096m  \
    -cp bin joshua.util.CompareGrammars \
    grammar1-file grammar2-file
\end{verbatim}


\section{Compatibility with Hiero}

In many cases it may be desirable to extract a grammar using Joshua that matches as closely as possible the grammars extracted by Hiero's grammar extraction code. The examples used in this section are taken directly from the unit tests for the ExtractRules class; for more information see the following files:

\begin{verbatim}
test/joshua/prefix_tree/ExtractRulesTest.java
src/joshua/prefix_tree/ExtractRules.java
src/joshua/prefix_tree/PrefixTree.java
\end{verbatim}


\subsection{Sentence-initial and sentence-final X}

One of the main points where Joshua's rule extraction code differs from Hiero's can be seen in two parameters of ExtractRules {\tt --sentence-initial-X} and {\tt --sentence-final-X}. Joshua currently defaults both of these parameters to true; to get a grammar that most closely matches Hiero's extracted grammars, these parameters should be set to true.

To see what happens here, consider the following parallel training corpus, with English as the source and German as the target:

\begin{alltt}
it makes him and it mars him , it sets him on yet it takes him off .
das macht ihn und es besch\"adigt ihn , es setzt ihn auf und es f\"uhrt ihn aus .
0-0 1-1 2-2 3-3 4-4 5-5 6-6 7-7 8-8 9-9 10-10 11-11 12-12 13-13 14-14 15-15 16-16 17-17
\end{alltt}

Now, what happens if we use Joshua's ExtractRules to extract a grammar for the following test sentence, with {\tt --sentence-initial-X=false} and {\tt --sentence-final-X=false}?

\begin{verbatim}
Test sentence:		it

Extracted rules:

"[X]", "it", "das"
"[X]", "it", "es"
\end{verbatim}

OK. Not too surprising. This is what you would probably expect given the training data. But, what happens if we use {\tt --sentence-initial-X=true} and {\tt --sentence-final-X=true}?

\begin{verbatim}
Test sentence:		it

Extracted rules:

"[X]", "[X,1] it [X,2]", "[X,1] es [X,2]"
"[X]", "[X,1] it", "[X,1] es"
"[X]", "it [X,1]", "das [X,1]"
"[X]", "it [X,1]", "es [X,1]"
"[X]", "it", "das"
"[X]", "it", "es"
\end{verbatim}

Now this is odd. We're now extracting rules with X at the beginning and end of rules, even when those nonterminals entirely precede or entirely follow all of the words in the test sentence. But, that's how Hiero's rule extraction behaves, so this option enables compatibility with that behavior.


\subsection{Violating phrase span limits}

The parameter {\tt --maxPhraseSpan}, which ExtractRules defaults to {\tt 10}, specifies the maximum span in the training corpus of any extracted source phrase. Consider the following sentence:

\begin{tt}
\ \\
declaro reanudado el per\'iodo de sesiones del parlamento europeo , interrumpido el viernes 17 de diciembre pasado , y reitero a sus se\~nor\'ias mi deseo de que hayan tenido unas buenas vacaciones .
\ \\
\end{tt}

During the prefix tree construction algrorithm of \newcite{lopez08phd}, phrases are built incrementally as continually larger phrases are examined.  A special case arises during rule extraction when considering phrases that start or end with a nonterminal, and the phrase span runs against the maximum phrase span constraint.  Given the preceding sentence and a phrase span limit of 10, the following phrase should certainly be allowed:

\begin{verbatim}
, X , y
\end{verbatim}

The above phrase spans exactly 10 words in the source sentence. But, should the phrase below be allowed:

\begin{verbatim}
, X , y X
\end{verbatim}

The phrase above, which has a span of (at minimum) 11, apparently violates the maximum phrase span constraint (at least under the most straightforward definition of maximum phrase span). A similar situation can occur with phrases with an initial nonterminal:

\begin{verbatim}
X , X que
\end{verbatim}

Joshua can be configured to allow or prohibit violation of the phrase span limit in these circumstances using the {\tt --violating-X} option in ExtractRules. Hiero's grammar extraction code allows these violations. For maximum compatibility with Hiero, ExtractRules defaults to {\tt --violating-X=true}. To change this behavior, set {\tt --violating-X=false} in ExtractRules.

